% Options for packages loaded elsewhere
\PassOptionsToPackage{unicode}{hyperref}
\PassOptionsToPackage{hyphens}{url}
%
\documentclass[
]{article}
\usepackage{lmodern}
\usepackage{amssymb,amsmath}
\usepackage{ifxetex,ifluatex}
\ifnum 0\ifxetex 1\fi\ifluatex 1\fi=0 % if pdftex
  \usepackage[T1]{fontenc}
  \usepackage[utf8]{inputenc}
  \usepackage{textcomp} % provide euro and other symbols
\else % if luatex or xetex
  \usepackage{unicode-math}
  \defaultfontfeatures{Scale=MatchLowercase}
  \defaultfontfeatures[\rmfamily]{Ligatures=TeX,Scale=1}
\fi
% Use upquote if available, for straight quotes in verbatim environments
\IfFileExists{upquote.sty}{\usepackage{upquote}}{}
\IfFileExists{microtype.sty}{% use microtype if available
  \usepackage[]{microtype}
  \UseMicrotypeSet[protrusion]{basicmath} % disable protrusion for tt fonts
}{}
\makeatletter
\@ifundefined{KOMAClassName}{% if non-KOMA class
  \IfFileExists{parskip.sty}{%
    \usepackage{parskip}
  }{% else
    \setlength{\parindent}{0pt}
    \setlength{\parskip}{6pt plus 2pt minus 1pt}}
}{% if KOMA class
  \KOMAoptions{parskip=half}}
\makeatother
\usepackage{xcolor}
\IfFileExists{xurl.sty}{\usepackage{xurl}}{} % add URL line breaks if available
\IfFileExists{bookmark.sty}{\usepackage{bookmark}}{\usepackage{hyperref}}
\hypersetup{
  pdftitle={lshin\_hw2},
  pdfauthor={Lucas Shin},
  hidelinks,
  pdfcreator={LaTeX via pandoc}}
\urlstyle{same} % disable monospaced font for URLs
\usepackage[margin=1in]{geometry}
\usepackage{color}
\usepackage{fancyvrb}
\newcommand{\VerbBar}{|}
\newcommand{\VERB}{\Verb[commandchars=\\\{\}]}
\DefineVerbatimEnvironment{Highlighting}{Verbatim}{commandchars=\\\{\}}
% Add ',fontsize=\small' for more characters per line
\usepackage{framed}
\definecolor{shadecolor}{RGB}{248,248,248}
\newenvironment{Shaded}{\begin{snugshade}}{\end{snugshade}}
\newcommand{\AlertTok}[1]{\textcolor[rgb]{0.94,0.16,0.16}{#1}}
\newcommand{\AnnotationTok}[1]{\textcolor[rgb]{0.56,0.35,0.01}{\textbf{\textit{#1}}}}
\newcommand{\AttributeTok}[1]{\textcolor[rgb]{0.77,0.63,0.00}{#1}}
\newcommand{\BaseNTok}[1]{\textcolor[rgb]{0.00,0.00,0.81}{#1}}
\newcommand{\BuiltInTok}[1]{#1}
\newcommand{\CharTok}[1]{\textcolor[rgb]{0.31,0.60,0.02}{#1}}
\newcommand{\CommentTok}[1]{\textcolor[rgb]{0.56,0.35,0.01}{\textit{#1}}}
\newcommand{\CommentVarTok}[1]{\textcolor[rgb]{0.56,0.35,0.01}{\textbf{\textit{#1}}}}
\newcommand{\ConstantTok}[1]{\textcolor[rgb]{0.00,0.00,0.00}{#1}}
\newcommand{\ControlFlowTok}[1]{\textcolor[rgb]{0.13,0.29,0.53}{\textbf{#1}}}
\newcommand{\DataTypeTok}[1]{\textcolor[rgb]{0.13,0.29,0.53}{#1}}
\newcommand{\DecValTok}[1]{\textcolor[rgb]{0.00,0.00,0.81}{#1}}
\newcommand{\DocumentationTok}[1]{\textcolor[rgb]{0.56,0.35,0.01}{\textbf{\textit{#1}}}}
\newcommand{\ErrorTok}[1]{\textcolor[rgb]{0.64,0.00,0.00}{\textbf{#1}}}
\newcommand{\ExtensionTok}[1]{#1}
\newcommand{\FloatTok}[1]{\textcolor[rgb]{0.00,0.00,0.81}{#1}}
\newcommand{\FunctionTok}[1]{\textcolor[rgb]{0.00,0.00,0.00}{#1}}
\newcommand{\ImportTok}[1]{#1}
\newcommand{\InformationTok}[1]{\textcolor[rgb]{0.56,0.35,0.01}{\textbf{\textit{#1}}}}
\newcommand{\KeywordTok}[1]{\textcolor[rgb]{0.13,0.29,0.53}{\textbf{#1}}}
\newcommand{\NormalTok}[1]{#1}
\newcommand{\OperatorTok}[1]{\textcolor[rgb]{0.81,0.36,0.00}{\textbf{#1}}}
\newcommand{\OtherTok}[1]{\textcolor[rgb]{0.56,0.35,0.01}{#1}}
\newcommand{\PreprocessorTok}[1]{\textcolor[rgb]{0.56,0.35,0.01}{\textit{#1}}}
\newcommand{\RegionMarkerTok}[1]{#1}
\newcommand{\SpecialCharTok}[1]{\textcolor[rgb]{0.00,0.00,0.00}{#1}}
\newcommand{\SpecialStringTok}[1]{\textcolor[rgb]{0.31,0.60,0.02}{#1}}
\newcommand{\StringTok}[1]{\textcolor[rgb]{0.31,0.60,0.02}{#1}}
\newcommand{\VariableTok}[1]{\textcolor[rgb]{0.00,0.00,0.00}{#1}}
\newcommand{\VerbatimStringTok}[1]{\textcolor[rgb]{0.31,0.60,0.02}{#1}}
\newcommand{\WarningTok}[1]{\textcolor[rgb]{0.56,0.35,0.01}{\textbf{\textit{#1}}}}
\usepackage{graphicx,grffile}
\makeatletter
\def\maxwidth{\ifdim\Gin@nat@width>\linewidth\linewidth\else\Gin@nat@width\fi}
\def\maxheight{\ifdim\Gin@nat@height>\textheight\textheight\else\Gin@nat@height\fi}
\makeatother
% Scale images if necessary, so that they will not overflow the page
% margins by default, and it is still possible to overwrite the defaults
% using explicit options in \includegraphics[width, height, ...]{}
\setkeys{Gin}{width=\maxwidth,height=\maxheight,keepaspectratio}
% Set default figure placement to htbp
\makeatletter
\def\fps@figure{htbp}
\makeatother
\setlength{\emergencystretch}{3em} % prevent overfull lines
\providecommand{\tightlist}{%
  \setlength{\itemsep}{0pt}\setlength{\parskip}{0pt}}
\setcounter{secnumdepth}{-\maxdimen} % remove section numbering

\title{lshin\_hw2}
\author{Lucas Shin}
\date{10/15/2020}

\begin{document}
\maketitle

\begin{Shaded}
\begin{Highlighting}[]
\KeywordTok{library}\NormalTok{(tidyverse)}
\end{Highlighting}
\end{Shaded}

\begin{verbatim}
## -- Attaching packages -------------------------------------------------------------------- tidyverse 1.3.0 --
\end{verbatim}

\begin{verbatim}
## v ggplot2 3.3.2     v purrr   0.3.4
## v tibble  3.0.3     v dplyr   1.0.2
## v tidyr   1.1.2     v stringr 1.4.0
## v readr   1.3.1     v forcats 0.5.0
\end{verbatim}

\begin{verbatim}
## -- Conflicts ----------------------------------------------------------------------- tidyverse_conflicts() --
## x dplyr::filter() masks stats::filter()
## x dplyr::lag()    masks stats::lag()
\end{verbatim}

\begin{Shaded}
\begin{Highlighting}[]
\KeywordTok{library}\NormalTok{(ggplot2)}
\KeywordTok{library}\NormalTok{(}\StringTok{"png"}\NormalTok{)}
\end{Highlighting}
\end{Shaded}

Deliverable \#1

\begin{Shaded}
\begin{Highlighting}[]
\CommentTok{# step 1, using 'croppedtraining' set of images}
\NormalTok{faces <-}\StringTok{ }\KeywordTok{matrix}\NormalTok{(}\DataTypeTok{nrow =} \DecValTok{14400}\NormalTok{, }
                         \DataTypeTok{ncol =} \DecValTok{0}\NormalTok{)}

\ControlFlowTok{for}\NormalTok{ (i }\ControlFlowTok{in} \DecValTok{1}\OperatorTok{:}\DecValTok{180}\NormalTok{) \{}
\NormalTok{   photo =}\StringTok{ }\KeywordTok{paste}\NormalTok{(}\KeywordTok{toString}\NormalTok{(i), }
                 \StringTok{"_0.png"}\NormalTok{, }
                 \DataTypeTok{sep =} \StringTok{""}\NormalTok{)}
\NormalTok{   image =}\StringTok{ }\KeywordTok{readPNG}\NormalTok{(}\KeywordTok{paste}\NormalTok{(}\StringTok{"/Users/lucasshin/Desktop/Biometrics/Biometrics HW 2/croppedtraining/"}\NormalTok{,}
\NormalTok{                         photo, }\DataTypeTok{sep =} \StringTok{""}\NormalTok{))}
\NormalTok{   vecImage =}\StringTok{ }\KeywordTok{as.vector}\NormalTok{(image)}
\NormalTok{   faces <-}\StringTok{ }\KeywordTok{cbind}\NormalTok{(faces, vecImage)}
\NormalTok{\}}

\NormalTok{rowMeans <-}\StringTok{ }\KeywordTok{rowMeans}\NormalTok{(faces)}

\CommentTok{# step 2 }
\CommentTok{# iterates through matrix and subtracts each row mean from each element in that row}
\ControlFlowTok{for}\NormalTok{ (j }\ControlFlowTok{in} \DecValTok{1}\OperatorTok{:}\KeywordTok{nrow}\NormalTok{(faces)) \{}
   \ControlFlowTok{for}\NormalTok{ (k }\ControlFlowTok{in} \DecValTok{1}\OperatorTok{:}\KeywordTok{ncol}\NormalTok{(faces)) \{}
\NormalTok{      faces[j,k] <-}\StringTok{ }\NormalTok{faces[j,k] }\OperatorTok{-}\StringTok{ }\NormalTok{rowMeans[j]}
\NormalTok{   \}}
\NormalTok{\}}

\CommentTok{# step 3}
\NormalTok{transposedMatrix <-}\StringTok{ }\KeywordTok{t}\NormalTok{(faces)}
\NormalTok{covMatrix <-}\StringTok{ }\NormalTok{faces }\OperatorTok\StringTok{ }\NormalTok{transposedMatrix}

\CommentTok{# step 4}
\NormalTok{ev <-}\StringTok{ }\KeywordTok{eigen}\NormalTok{(covMatrix)}
\NormalTok{eigenVec <-}\StringTok{ }\NormalTok{ev}\OperatorTok{$}\NormalTok{vectors}
\NormalTok{eigenVal <-}\StringTok{ }\NormalTok{ev}\OperatorTok{$}\NormalTok{values}

\CommentTok{# step 5}

\CommentTok{# to find weighted sum using all eigenvectors}
\NormalTok{inputImage <-}\StringTok{ }\NormalTok{faces[,}\DecValTok{1}\NormalTok{]}
\NormalTok{weight1 <-}\StringTok{ }\KeywordTok{abs}\NormalTok{(}\KeywordTok{t}\NormalTok{(eigenVec) }\OperatorTok\StringTok{ }\NormalTok{inputImage)}
\NormalTok{weight1Sum <-}\StringTok{ }\KeywordTok{colSums}\NormalTok{(weight1)}

\CommentTok{# to find weighted sum using a subset (first 10) of the eigenvectors}
\NormalTok{evSubset <-}\StringTok{ }\NormalTok{eigenVec[,}\DecValTok{1}\OperatorTok{:}\DecValTok{10}\NormalTok{]}
\NormalTok{weightSub <-}\StringTok{ }\KeywordTok{abs}\NormalTok{(}\KeywordTok{t}\NormalTok{(evSubset) }\OperatorTok\StringTok{ }\NormalTok{inputImage)}
\NormalTok{weightSubSum <-}\StringTok{ }\KeywordTok{colSums}\NormalTok{(weightSub)}
\end{Highlighting}
\end{Shaded}

Deliverable (2)

\begin{Shaded}
\begin{Highlighting}[]
\NormalTok{gen1a <-}\StringTok{ "/Users/lucasshin/Desktop/Biometrics/Biometrics HW 2/testingfaces/s26d10.png"}
\NormalTok{gen1b <-}\StringTok{ "/Users/lucasshin/Desktop/Biometrics/Biometrics HW 2/testingfaces/s26d7.png"}

\NormalTok{gen2a <-}\StringTok{ "/Users/lucasshin/Desktop/Biometrics/Biometrics HW 2/testingfaces/s24d9.png"}
\NormalTok{gen2b <-}\StringTok{ "/Users/lucasshin/Desktop/Biometrics/Biometrics HW 2/testingfaces/s24d1.png"}

\NormalTok{gen3a <-}\StringTok{ "/Users/lucasshin/Desktop/Biometrics/Biometrics HW 2/testingfaces/s17d2.png"}
\NormalTok{gen3b <-}\StringTok{ "/Users/lucasshin/Desktop/Biometrics/Biometrics HW 2/testingfaces/s17d8.png"}

\NormalTok{imp1a <-}\StringTok{ "/Users/lucasshin/Desktop/Biometrics/Biometrics HW 2/testingfaces/s13d9.png"}
\NormalTok{imp1b <-}\StringTok{ "/Users/lucasshin/Desktop/Biometrics/Biometrics HW 2/testingfaces/s12d10.png"}

\NormalTok{imp2a <-}\StringTok{ "/Users/lucasshin/Desktop/Biometrics/Biometrics HW 2/testingfaces/s05d3.png"}
\NormalTok{imp2b <-}\StringTok{ "/Users/lucasshin/Desktop/Biometrics/Biometrics HW 2/testingfaces/s02d4.png"}

\NormalTok{imp3a <-}\StringTok{ "/Users/lucasshin/Desktop/Biometrics/Biometrics HW 2/testingfaces/s25d1.png"}
\NormalTok{imp3b <-}\StringTok{ "/Users/lucasshin/Desktop/Biometrics/Biometrics HW 2/testingfaces/s13d7.png"}

\NormalTok{testing <-}\StringTok{ }\KeywordTok{c}\NormalTok{(gen1a, gen1b, gen2a, gen2b, gen3a, gen3b, imp1a, imp1b, imp2a, imp2b, imp3a, imp3b)}

\NormalTok{testfaces <-}\StringTok{ }\KeywordTok{matrix}\NormalTok{(}\DataTypeTok{nrow =} \DecValTok{14400}\NormalTok{, }\DataTypeTok{ncol =} \DecValTok{0}\NormalTok{)}

\ControlFlowTok{for}\NormalTok{ (h }\ControlFlowTok{in} \DecValTok{1}\OperatorTok{:}\KeywordTok{length}\NormalTok{(testing)) \{}
\NormalTok{   image =}\StringTok{ }\NormalTok{testing[h]}
\NormalTok{   vecImage =}\StringTok{ }\KeywordTok{as.vector}\NormalTok{(}\KeywordTok{readPNG}\NormalTok{(image))}
\NormalTok{   testfaces <-}\StringTok{ }\KeywordTok{cbind}\NormalTok{(testfaces, vecImage)}
\NormalTok{\}}

\NormalTok{rowMeans <-}\StringTok{ }\KeywordTok{rowMeans}\NormalTok{(testfaces)}

\CommentTok{# iterates through matrix and subtracts each row mean from each element in that row}
\ControlFlowTok{for}\NormalTok{ (j }\ControlFlowTok{in} \DecValTok{1}\OperatorTok{:}\KeywordTok{nrow}\NormalTok{(testfaces)) \{}
   \ControlFlowTok{for}\NormalTok{ (k }\ControlFlowTok{in} \DecValTok{1}\OperatorTok{:}\KeywordTok{ncol}\NormalTok{(testfaces)) \{}
\NormalTok{      testfaces[j,k] <-}\StringTok{ }\NormalTok{testfaces[j,k] }\OperatorTok{-}\StringTok{ }\NormalTok{rowMeans[j]}
\NormalTok{   \}}
\NormalTok{\}}

\CommentTok{## comparisons}
\NormalTok{input1 <-}\StringTok{ }\NormalTok{testfaces[,}\DecValTok{1}\NormalTok{]}
\NormalTok{input2 <-}\StringTok{ }\NormalTok{testfaces[,}\DecValTok{2}\NormalTok{]}
\NormalTok{weight1 <-}\StringTok{ }\KeywordTok{abs}\NormalTok{(}\KeywordTok{t}\NormalTok{(evSubset) }\OperatorTok\StringTok{ }\NormalTok{input1)}
\NormalTok{weight2 <-}\StringTok{ }\KeywordTok{abs}\NormalTok{(}\KeywordTok{t}\NormalTok{(evSubset) }\OperatorTok\StringTok{ }\NormalTok{input2)}
\NormalTok{distance1 <-}\StringTok{ }\KeywordTok{abs}\NormalTok{(weight1}\OperatorTok{-}\NormalTok{weight2)}

\NormalTok{input3 <-}\StringTok{ }\NormalTok{testfaces[,}\DecValTok{3}\NormalTok{]}
\NormalTok{input4 <-}\StringTok{ }\NormalTok{testfaces[,}\DecValTok{4}\NormalTok{]}
\NormalTok{weight3 <-}\StringTok{ }\KeywordTok{abs}\NormalTok{(}\KeywordTok{t}\NormalTok{(evSubset) }\OperatorTok\StringTok{ }\NormalTok{input3)}
\NormalTok{weight4 <-}\StringTok{ }\KeywordTok{abs}\NormalTok{(}\KeywordTok{t}\NormalTok{(evSubset) }\OperatorTok\StringTok{ }\NormalTok{input4)}
\NormalTok{distance2 <-}\StringTok{ }\KeywordTok{abs}\NormalTok{(weight3}\OperatorTok{-}\NormalTok{weight4)}

\NormalTok{input5 <-}\StringTok{ }\NormalTok{testfaces[,}\DecValTok{5}\NormalTok{]}
\NormalTok{input6 <-}\StringTok{ }\NormalTok{testfaces[,}\DecValTok{6}\NormalTok{]}
\NormalTok{weight5 <-}\StringTok{ }\KeywordTok{abs}\NormalTok{(}\KeywordTok{t}\NormalTok{(evSubset) }\OperatorTok\StringTok{ }\NormalTok{input5)}
\NormalTok{weight6 <-}\StringTok{ }\KeywordTok{abs}\NormalTok{(}\KeywordTok{t}\NormalTok{(evSubset) }\OperatorTok\StringTok{ }\NormalTok{input6)}
\NormalTok{distance3 <-}\StringTok{ }\KeywordTok{abs}\NormalTok{(weight5}\OperatorTok{-}\NormalTok{weight6)}

\NormalTok{input7 <-}\StringTok{ }\NormalTok{testfaces[,}\DecValTok{7}\NormalTok{]}
\NormalTok{input8 <-}\StringTok{ }\NormalTok{testfaces[,}\DecValTok{8}\NormalTok{]}
\NormalTok{weight7 <-}\StringTok{ }\KeywordTok{abs}\NormalTok{(}\KeywordTok{t}\NormalTok{(evSubset) }\OperatorTok\StringTok{ }\NormalTok{input7)}
\NormalTok{weight8 <-}\StringTok{ }\KeywordTok{abs}\NormalTok{(}\KeywordTok{t}\NormalTok{(evSubset) }\OperatorTok\StringTok{ }\NormalTok{input8)}
\NormalTok{distance4 <-}\StringTok{ }\KeywordTok{abs}\NormalTok{(weight7}\OperatorTok{-}\NormalTok{weight8)}

\NormalTok{input9 <-}\StringTok{ }\NormalTok{testfaces[,}\DecValTok{9}\NormalTok{]}
\NormalTok{input10 <-}\StringTok{ }\NormalTok{testfaces[,}\DecValTok{10}\NormalTok{]}
\NormalTok{weight9 <-}\StringTok{ }\KeywordTok{abs}\NormalTok{(}\KeywordTok{t}\NormalTok{(evSubset) }\OperatorTok\StringTok{ }\NormalTok{input9)}
\NormalTok{weight10 <-}\StringTok{ }\KeywordTok{abs}\NormalTok{(}\KeywordTok{t}\NormalTok{(evSubset) }\OperatorTok\StringTok{ }\NormalTok{input10)}
\NormalTok{distance5 <-}\StringTok{ }\KeywordTok{abs}\NormalTok{(weight9}\OperatorTok{-}\NormalTok{weight10)}

\NormalTok{input11 <-}\StringTok{ }\NormalTok{testfaces[,}\DecValTok{11}\NormalTok{]}
\NormalTok{input12 <-}\StringTok{ }\NormalTok{testfaces[,}\DecValTok{12}\NormalTok{]}
\NormalTok{weight11 <-}\StringTok{ }\KeywordTok{abs}\NormalTok{(}\KeywordTok{t}\NormalTok{(evSubset) }\OperatorTok\StringTok{ }\NormalTok{input11)}
\NormalTok{weight12 <-}\StringTok{ }\KeywordTok{abs}\NormalTok{(}\KeywordTok{t}\NormalTok{(evSubset) }\OperatorTok\StringTok{ }\NormalTok{input12)}
\NormalTok{distance6 <-}\StringTok{ }\KeywordTok{abs}\NormalTok{(weight11}\OperatorTok{-}\NormalTok{weight12)}
\end{Highlighting}
\end{Shaded}

\end{document}
